\documentclass[a4paper]{article}
\setlength{\topmargin}{-25mm}
\setlength{\textwidth}{7in}
\setlength{\oddsidemargin}{-8mm}
\setlength{\textheight}{10in}
\setlength{\footskip}{1in}
\setlength{\parindent}{0in}
\usepackage{color,hyperref}
\definecolor{darkblue}{rgb}{0.0,0.0,0.3}
\hypersetup{colorlinks,breaklinks,
            linkcolor=darkblue,urlcolor=darkblue,
            anchorcolor=darkblue,citecolor=darkblue}

\begin{document}
\today \\
\\
Feingold Technologies GmbH, \newline
Berlin, \newline
Germany \newline

Dear Madam/Sir,
\\
\\
I am writing to apply for the position of data scientist at Feingold Technologies GmbH. I am graduating during the summer of 2018 in computer science with a specialization in machine learning from the University of Freiburg, Germany. I want to pursue a career in applied machine learning.
\\
\\
Currently for my thesis, I compute similarities in graphical and text data to build a recommendation system. To achieve that, I use a combination of text mining, matrix factorization (latent semantic indexing), neural network, and optimization techniques. The matrix factorization technique which reduces the rank of a matrix works well to learn the hidden relations in paragraph-token matrix but learning these embeddings using neural network helps me achieve a better similarity among paragraphs. For the multiple sources of text/tokens, I learn importance weights using gradient descent on these sources in order to combine them to get a better similarity score. Further, for predicting nodes in a directed graph, I use neural networks. It takes in the stream of input nodes as vectors to predict the next node. After processing the raw graphs, the problem becomes a classification one. I consider multiple ways like hot-vectors, word2vec and node2vec to create node embeddings (vectors). I develop this system in Python.
\\
\\
I finished a project in brain-signals classification which generates artificial brain signals using the original ones by doing mathematical transformations. In order to reduce overfitting due to less number of original brain signals, I used a regularized variant of linear discriminant analysis for classification. I love to contribute to open-source software, 
\href{https://usegalaxy.org/} {Galaxy}, for my research assistant work.
\\
\\
I want to be the part of a small team which leads to a quality personal interaction with coworkers and provides a conducive environment for better learning. Hence, the job position will provide me an opportunity to improve my professional and personal skills along with the contribution to the organisation.
\\
\\
Thank you for your consideration. I look forward to hearing from you.
\\
\\
Sincerely, \\
Anup Kumar \newline
\\
\\
\\
\\
Department of Computer Science, \\
University of Freiburg, \\
Freiburg 79110, Germany \\
anup.kumar@uranus.uni-freiburg.de\\
anup.rulez@gmail.com\\
+49-1522-6169-642\\


\end{document}